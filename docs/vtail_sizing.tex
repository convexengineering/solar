\documentclass[10pt, a4paper]{article}
\usepackage{latexsym}
\usepackage{amssymb,amsmath}
\usepackage[pdftex]{graphicx}
\newcommand{\dbar}[1]{\Bar{\Bar{#1}}}

\topmargin = 0.1in \textwidth=5.7in \textheight=8.6in

\oddsidemargin = 0.1in \evensidemargin = 0.1in

% headers
\usepackage{fancyhdr}
\pagestyle{fancy}
\chead{} 
\rhead{\thepage} 
% footer
\lfoot{\small\scshape } 
\cfoot{} 
%%%% insert your name here %%%%
\rfoot{\footnotesize Author} 
\renewcommand{\headrulewidth}{.3pt} 
\renewcommand{\footrulewidth}{.3pt}
\setlength\voffset{-0.25in}
\setlength\textheight{648pt}

\begin{document}

\title{Vertical Tail Sizing by Volume Coefficient}
\author{Michael Burton}
\maketitle

For a very high aspect ratio aicraft, the vertical tail sizing may be sized by an adverse yaw requirement.  Looking at the stability derivatives, the maximum yaw force can be estimated

\begin{align}
    C_{n_p} &= \frac{C_{L_0}}{8} \\
    \frac{\partial N}{\partial \bar{p}} &= C_{n_p} Q Sb \\
    N_{\mathrm{max}} &= C_{n_p} Q S b \bar{p}_{\mathrm{max}}
\end{align}

The maximum moment from the rudder deflection must be at least equal to maximum yaw force

\begin{align}
    N_{\delta_{r_{\max}}} &= N_{\mathrm{max}} \\
    N_{\delta_{r_{\max}}} &= C_{L_{V_{\mathrm{max}}}} Q S_{\mathrm{v}} l_{\mathrm{v}}
\end{align}

Which reduces to 

\begin{align}
    C_{L_{V_{\max}}} Q S_{\mathrm{v}} l_{\mathrm{v}} &= C_{n_p} Q S b \bar{p}_{\max} \\
    \frac{S_{\mathrm{v}} l_{\mathrm{v}}}{Sb} &= \frac{C_{n_p} \bar{p}_{\max}}{C_{L_{V_{\max}}}} \\
    V_{\mathrm{v}} &= \frac{C_{L_0} \bar{p}_{\max}}{8 C_{L_{V_{\max}}}}
\end{align}

Let us assume the following values:

\begin{align}
    C_{L_0} &= 1.3 \\
    \bar{p}_{\max} &= 0.1 \\
    C_{L_{V_{\max}}} &= 1.0 
\end{align}

This follows that $V_{\mathrm{v}} = 0.015$, which would be a requirement on the vertical tail size.  

\end{document}
