% `advanced_example.tex', an advanced example employing the AIAA class
% plus other third-party LaTeX packages.
%
% For a bare-bones usage, see `template.tex'.
%
% Typical processing for PostScript (PS) output:
%
%  latex advanced_example
%  bibtex advanced_example  (bibliography)
%  makeindex -s nomencl.ist -o advanced_example.gls advanced_example.glo
%                            (nomenclature)
%  latex advanced_example   (repeat as needed to resolve references)
%
%  xdvi advanced_example    (onscreen draft display)
%  dvips advanced_example   (postscript)
%  gv advanced_example.ps   (onscreen display)
%  lpr advanced_example.ps  (hardcopy)
%
% With the above, only Encapsulated PostScript (EPS) images can be used.
%
%
%  pdflatex advanced_example
%  bibtex advanced_example    (bibliography)
%  makeindex -s nomencl.ist -o advanced_example.gls advanced_example.glo
%                              (nomenclature)
%  pdflatex advanced_example  (repeat as needed to resolve references)
%
%  acroread advanced_example.pdf  (onscreen display)
%
% If you have EPS figures, you will need to use the epstopdf script
% to convert them to PDF because PDF is a limmited subset of EPS.
% pdflatex accepts a variety of other image formats such as JPG, TIFF,
% PNG, and so forth -- check the documentation for your version.
%
% If you do *not* specify suffixes when using the graphicx package's
% \includegraphics command, latex and pdflatex will automatically select
% the appropriate figure format from those available.  This allows you
% to produce PS and PDF output from the same LaTeX source file.
%
% To generate a large format (e.g., 11"x17") PostScript copy for editing
% purposes, use
%
%  dvips -x 1467 -O -0.65in,0.85in -t tabloid advanced_example
%
% For further details and support, read the Users Manual, aiaa.pdf.

\documentclass[]{aiaa-tc}% insert '[draft]' option to show overfull boxes

 \usepackage{varioref}%  smart page, figure, table, and equation referencing
 \usepackage{wrapfig}%   wrap figures/tables in text (i.e., Di Vinci style)
 \usepackage{threeparttable}% tables with footnotes
 \usepackage{dcolumn}%   decimal-aligned tabular math columns
  \newcolumntype{d}{D{.}{.}{-1}}
 \usepackage{nomencl}%   nomenclature generation via makeindex
  \makeglossary
 \usepackage{amssymb,amsmath}
 \usepackage{subfigure}% subcaptions for subfigures
 \usepackage{subfigmat}% matrices of similar subfigures, aka small mulitples
 \usepackage{fancyvrb}%  extended verbatim environments
 \fvset{fontsize=\footnotesize,xleftmargin=2em}
 \usepackage{lettrine}%  dropped capital letter at beginning of paragraph
%  \usepackage[dvips]{dropping}% alternative dropped capital package
 \usepackage[colorlinks]{hyperref}%  hyperlinks [must be loaded after dropping]
 \usepackage{float}
 \usepackage{longtable,booktabs,tabularx}
 % \restylefloat{table}
 \usepackage{graphicx}
 \usepackage{caption}
 \usepackage{siunitx}
 \usepackage{multicol}
 \usepackage{indentfirst}
 \usepackage{environ}
 \usepackage[labelfont=bf]{caption}
 \usepackage{multirow}
 \usepackage{setspace}
%  \graphicspath{{./figs/}}
 \usepackage[sort, numbers]{natbib}

\usepackage{bm}
\title{Solar Aicraft Design Trade Studies Using Geometric Programming}

 \author{
  Mark Drela\thanks{Terry J. Kohler Professor, Aeronautics and Astronautics Engineering, MIT, 77 Mass Ave, Cambridge MA, 02139, AIAA Member.}, 
  Michael Burton\thanks{Ph.D. Student, Aeronautics and Astronautics Engineering, MIT, 77 Mass Ave, Cambridge MA, 02139, AIAA Student.}, 
  Dorian Colas\thanks{Airframe Design Lead, Facebook, 1 Hacker Way, Menlo Park CA, 94025}, 
  Nicholas Roberts\thanks{Airframe Optimization Engineer, Facebook, 1 Hacker Way, Menlo Park CA, 94025}, 
  Vishvas Samuel\thanks{Airframe Structural Engineer, Facebook, 1 Hacker Way, Menlo Park CA, 94025}\\
  {\normalsize\itshape
   Massachusetts Institute of Technology, Cambridge, 02139, USA}\\
 }

 % Data used by 'handcarry' option
 \AIAApapernumber{YEAR-NUMBER}
 \AIAAconference{Conference Name, Date, and Location}
 \AIAAcopyright{\AIAAcopyrightD{YEAR}}

 % Define commands to assure consistent treatment throughout document
 \newcommand{\eqnref}[1]{(\ref{#1})}
 \newcommand{\class}[1]{\texttt{#1}}
 \newcommand{\package}[1]{\texttt{#1}}
 \newcommand{\file}[1]{\texttt{#1}}
 \newcommand{\BibTeX}{\textsc{Bib}\TeX}
 \usepackage{hyperref}
 \hypersetup{citecolor = blue}

\begin{document}

\graphicspath{{./figs/}} 
\maketitle

\begin{abstract}
    Solar aircraft represents a highly coupled, multidisciplinary problem. 
    Trade-offs between aerodynamic performance, structural integrity, component performance, cost, and manufacturability are non-intuitive. 
    Using geometric programming configuration trades, resizing analysis, and design parameter sensitivity can all be achieved to a level of speed and reliability not previously achieved in conceptual and preliminary aircraft design. 
    Using this optimization method trade studies in design parameters, configurations and requirements are presented for a solar-electric aircraft.  
    The results of these trade studies are used to inform design decisions and validate higher order models. 
\end{abstract}

\section*{Nomenclature}

\begin{multicols}{2}
\small

\begin{tabbing}
  XXXXXXX \= \kill% this line sets tab stop
$A$ \> wing aspect ratio \\
$b$ \> wing span \\ % [ft] \\
$c$ \> wing chord \\ %[m] \\
$C_f$ \> skin friction coefficient \\
$C_L$ \> lift coefficient \\
$C_{L_0}$ \> zero alpha lift coefficient \\
$C_{L_h}$ \> horizontal tail lift coefficient \\
$C_{L_{h_0}}$ \> zero alpha horizontal tail lift coefficient \\
$C_m$ \> moment coefficient \\ 
$D_{\text{fuse}}$ \> fuselage drag \\% [N] \\
$\Delta W$ \> wing section weight \\ %[N] \\
$\Delta y$ \> wing section length \\ %[m] \\
$e$ \> span efficiency factor \\
$E$ \> Young's Modulus \\ %[Pa] \\
$E_{\text{batt}}$ \> energy stored in battery \\ %[J] \\
$g$ \> gravitational constant \\ % [m/s$^2$] \\
$h$ \> flight altitude \\ % [ft] \\
$h_{\text{batt}}$ \> battery specific energy \\ % [Whr/kg] \\
$h_{\text{cap}}$ \> spar cap separation \\ % [m] \\
$I$ \> cap spar moment of inertia \\ % [m$^4$] \\
$I_0$ \> tail boom root moment of inertia \\ % [m$^4$] \\
$k$ \> tail boom taper index \\
$k_{\mathrm{fuse}}$ \> fuselage form factor \\
$L_\text{h}$ \> horizontal tail lift \\ % [N] \\
$L_\text{w}$ \> wing lift \\ % [N] \\
$l_\text{fuse}$ \> fuselage length \\ % [m] \\
$l_\text{h}$ \> horizontal tail moment arm \\ % [m] \\
$M_{\mathrm{cg}}$ \> moment about center of mass \\
$M_{\mathrm{w}}$ \> wing moment \\
$N_{\text{g}}$ \> g-loading factor\\
$N_{\text{gust}}$ \> gust load factor\\
$R_{\text{fuse}}$ \> fuselage radius \\ % [m] \\
$Re$ \> Reynolds number \\
$S$ \> wing planform area \\ % [m$^2$]\\
$S_{\text{fuse}}$ \> fuselage wetted surface area \\ % [m$^2$]\\
$S_{\text{h}}$ \> horizontal tail planform area \\ % [m$^2$]\\
SM$_{\mathrm{min}}$ \> minimum static margin \\
$t_{\text{cap}}$ \> spar cap thickness \\ % [m] \\
$t_{\text{core}}$ \> spar core thickness \\ % [m] \\
$t_{\text{shear}}$ \> shear web thickness \\
$V$ \> true airspeed \\ % [m/s] \\
$\mathcal{V}_{\text{fuse}}$ \>  fuselage volume \\ % [m$^3$] \\
$V_{\text{h}}$ \> horizontal tail lift coefficient \\
$W_{\text{batt}}$ \> battery weight \\ % [N] \\
$w_{\text{cap}}$ \> spar cap width \\ % [m] \\
$W_{\text{fuse}}$ \> fuselage weight \\ % [N] \\
$W_{\text{h}}$ \> horizontal tail weight \\ % [N] \\
$w_{\text{max}}$ \> maximum deflection limit \\ % [m] \\
$W_{\text{MTO}}$\> max take-off weight \\ % [N] \\
$W_{\text{pay}}$ \> payload weight \\ % [N] \\
$W_{\text{spar}}$ \> wing spar weight \\ % [N] \\
$W_{\text{wing}}$ \> wing weight \\ % [N] \\
$x_{\mathrm{cg}}$ \> center of gravity \\
$x_{\mathrm{ac}}$ \> aerodynamic center \\
$\alpha$ \> angle of attack \\
$\epsilon$ \> downwash angle \\
$\eta_{\text{charge}}$ \> battery charging efficiency \\
$\eta_{\text{discharge}}$ \> battery discharging efficiency \\
$\eta_{\text{prop}}$ \> propulsive efficiency \\
$\eta_{\text{solar}}$ \> solar cell efficiency \\
$\rho_{A_{\text{cfrp}}}$ \> area density of carbon fiber \\ % [kg/cm$^2$] \\
$\rho_{\text{cfrp}}$ \> density of carbon fiber \\ % [kg/cm$^3$] \\
$\rho_{\text{foam}}$ \> density of foam \\ % [lbf/ft$^3$] \\
$\sigma_{\text{cfrp}}$ \> maximum carbon fiber stress \\ % [Pa] \\
$\tau_t$ \> wing thickness to chord ratio \\
$\tau_w$ \> cap spar width to chord ratio \\
 \end{tabbing}

\end{multicols}
% \printglossary% creates nomenclature section produced by MakeIndex

\section{Introduction}

Multiple solar-electric, high-altitude, long-endurance (HALE) aircraft have been designed or built by NASA\cite{dunbar_2015}, Airbus\cite{zephyr}, Aurora Flight Sciences\cite{odysseus}, and Facebook\cite{aquila}.
Configurations among these solar aircraft vary widely in part because analyzing different configurations in an efficient and reliable way is difficult due to the multifaceted interaction between aerodynamics, structural weight, solar energy, and environmental factors.\cite{solartech}
For the same reason, it is difficult to identify key design parameters.  
This ambiguity makes down-selecting critical components such as batteries or solar cells, or correctly designing components to have the highest system level impact is potentially not obvious.  

Using the model developed by Burton and Hoburg\cite{burton_solar_2017}, this paper explores how geometric programming can be used in tandem with higher fidelity models to explore configuration, key design parameters, and requirement tradeoffs for solar HALE aircraft.
Geometric programming (GP) is chosen as a means of exploring this trade space because of its speed and reliability.  
Because GPs can solve for thousands of variables in seconds\cite{gp}, designs and configurations can be validated and analyzed efficiently.  
Additionally, design sensitivities to every parameter are an output of a GP solution, allowing for immediate identification of driving design parameters. 
While the form of GP often requires higher fidelity models to be approximated as convex constraints\cite{gpkit}, the analysis done through GP models captures important design trends and variables that can guide conceptual and preliminary design. 
Trade studies done in GP can narrow the design space, allowing faster progression through conceptual and preliminary design with higher fidelity methods. 
 
Solar HALE aircraft are typically designed to fly for months at a time.\cite{aquila}  
The constraining design case for year-round availability is completing a 24-hour cycle during the winter solstice when solar flux is at a minimum, relying on solar power during the day to operate and recharge batteries and running on battery power at night. 
If this can be achieved, then it can theoretically fly indefinitely.  
As a station keeping aircraft, it must fly faster than the local wind speeds. 
Solar flux and wind speed depend on latitude, season and percentile wind speed, which are defined as mission requirements.

Basic mission requirements are defined to size a solar aircraft using the GP model described by Burton and Hoburg.\cite{burton_solar_2017}
Different component and aircraft configurations are modeled in a GP form and evaluated against the baseline sizing. 
Sensitivities are analyzed to determine key design parameters. 
The sensitivities are also used show how improvements in solar cell and battery technology would reduce the weight and size of the aircraft. 
The season and latitude requirements are varied to consider families of aircraft for different requirements.  

This paper also demonstrates that higher fidelity models can be written in a GP form and used to capture important design trends.  
A tail boom flexibility model is presented to show how this can be accomplished.  
Future higher fidelity models may include a vortex lattice method, a propeller performance model, and battery and solar cell performance models. 

\section{Hypothetical Baseline Design}

A hypothetical sizing study for a solar HALE aircraft is achieved using the GP model defined in Burton and Hoburg.\cite{burton_solar_2017} 
The sizing is based on requirements of station keeping, seasonal availibility and latitude. 
To station keep, the aircraft must fly faster than the loacl wind speeds.  
Thus the station keeping requirement is defined as a percentage of the wind speeds per latitude.  
The seasonal requirement is defined as the capability to complete a mission during a given portion of the year.  
The latitude requirement is defined as the ability to fly anywhere between a band of latitudes. 
The nominal requirements are defined in Table~\ref{t:mreqs}.

\begin{longtable}{lccccccccccccc}
\caption{Mission Requirements}\\
\toprule
\toprule
\label{t:mreqs}
Requirement & Value \\ \hline
Station Keeping & 90\% winds \\
Season & all seasons\\
Latitude & $\pm25^{\circ}$\\
\bottomrule
\end{longtable}

Basic assumptions of the model described by Burton and Hoburg are summarized as: 

\begin{itemize}
    \item Configuration: The GP model assumes a constant tapered wing and conventional tail configuration.  The batteries are placed inside the wing to achieve span loading.  Solar panels are placed on the wing.  
    \item Operations: The aircraft is assumed to fly above 50,000 ft to avoid cloud coverage and to reach the local minimum wind speeds that occur around 60,000 ft.\cite{burton_solar_2017} The model trades air density with wind speed to achieve the optimum altitude. Solar flux is calculated as a function of latitude and day of the year and assumes the wing is flat. 
    \item Structural: The aircraft is subjected to two load cases: a standard g-loading and a one-minus-cosine distributed gust load.  The wing spar is assumed to take all of the bending loads and is a cap spar with foam core. The tail boom is a linearly tapered tube.  All materials are carbon fiber and foam.  
    \end{itemize}

    For the given design parameters in Table~\ref{t:params}, and with an objective function $\min(W_{\mathrm{MTO}})$, the model solved for 186 variables in 0.0654 seconds on a standard desktop computer. 
Table~\ref{t:design} shows various design variables of the output solution. 

\begin{multicols}{2}

\begin{table}[H]
    \centering
    \caption{Design Parameters}
    \label{t:params}
    \begin{tabular}{l c}
    \toprule
    \toprule
    Parameter                                   & Value \\ \hline
    $W_{\mathrm{pay}}$                          & 10 [lbs] \\
    $\eta_{\mathrm{charge, discharge}}$         & 0.98\\
    $h_{\mathrm{batt}}$                         & 350 [Whr/kg]\\
    $\eta_{\mathrm{solar}}$                     & 0.22 \\
    $N_g$                                       & 5 \\
    $N_{\mathrm{gust}}$                         & 2 \\
    $e$                                         & 0.95 \\
    $\eta_{\mathrm{prop}}$                      & 0.8  \\
    \bottomrule
\end{tabular}
\end{table}
    
\begin{table}[H]
    \centering
    \caption{Design Variables}
    \label{t:design}
    \begin{tabular}{l c}
    \toprule
    \toprule
    Variable                    & Value         \\ \hline
    $W_{\mathrm{MTOW}}$         & 176.2 [lbs]   \\
    $W_{\mathrm{batt}}$         & 75.1 [lbs]    \\
    $W_{\mathrm{wing}}$         & 46.1 [lbs]    \\
    $AR$                        & 25.5          \\
    $b$                         & 63 [ft]       \\
    $V$                         & 26.15 [ft]    \\
    $h$                         & 55,000 [ft]   \\
    $C_L$                       & 1.083         \\
    \bottomrule
\end{tabular}
\end{table}

\end{multicols}

One benefit of GP models is that any design parameter can be changed and the model resolved in a fraction of a second.  
Table~\ref{t:params} shows a non-comprehensive list of design parameters that can be changed, each change resulting in a newly sized aircraft. 
For that reason, the design parameter values, while representative of current technology, were chosen arbitrarily to be used as a point of comparison.

\section{Sensitivity Analysis}

In geometric programming the sensitivity to a parameter is defined as the percentage change in the objective function for a corresponding one percent change in that parameter value.  
For example, the sensitivity to battery specific energy $h_{\mathrm{batt}}$, is -4.74.  If the battery specific energy were increased by 1\% this would result in an approximate 4.74\% decrease in the total weight of the aircraft.  
Sensitivities are local approximations but are useful for understanding relative parameter importance.  
Figure~\ref{f:sensbar} shows the highest sensitivities for the design parameters described in the previous section. 

\begin{figure}[h!]
	\begin{center}
	\includegraphics[width=0.6\textwidth]{sensbar.pdf}
    \caption{\textbf{Parameters with highest sensitivity have largest effect on objective function. }}
	\label{f:sensbar}
	\end{center}
\end{figure}


These sensitivities can help identify the most important aspects of a design. 
For the given assumptions in this design, the propeller efficiency, battery specific energy and solar cell efficiency all have high sensitivities. 
This would suggest that higher fidelity models should be used to estimate the performance of these components to meet the design requirements. 
Note the sensitivity to the payload weight is relatively small, suggesting that efforts to reduce payload weight will have diminishing returns. 

Another useful application for sensitivities is to evaluate return on investment costs of improving technology.  
Batteries and solar cells both have high costs, but it is not immediately obvious which one, if improved, will have greater impact on aircraft size.  
If it is assumed that the cost of improving solar cells and the cost of improving batteries for a giver percentage are the same, then by using the sensitivities it becomes clear that investing in battery technology will have a greater impact on the design than by investing in solar cells. 

\section{Configuration Trade Studies}

\subsection{Spar Configuration}

This section explores how different wing spar configurations can be evaluated using GP. 
Two spar configurations are compared: a cap spar and a box spar.  

The baseline GP model previously mentioned assumes a cap spar configuration as the main structural element of the wing.\cite{burton_solar_2017} 
A cap spar has two unidirectional carbon fiber spanwise spar caps separated by a foam core and wrapped by a shear web as shown in Figure~\ref{f:capspar}. 
This a common spar configuration for carbon fiber wings.  


\begin{figure}[h!]
	\begin{center}
	\includegraphics[width=0.9\textwidth]{capspar.pdf}
    \caption{\textbf{Cross sectional view of a cap spar.}}
	\label{f:capspar}
	\end{center}
\end{figure}

Considering only the carbon fiber for bending, the equation for the moment of inertia\cite{bending} of a cap spar can be expressed as

\begin{equation}
    \label{e:moispar}
    I_i = \frac{w_{\text{cap}_i}t_{\text{cap}_i}^3}{6} + 2w_{\text{cap}_i}t_{\text{cap}_i}\left( \frac{h_{\text{cap}_i}}{2} + \frac{t_{\text{cap}_i}}{2} \right)^2.
\end{equation}

This equation is not GP compatible.  However, using a first order conservative approximation, the moment of inertia can be simplified to be written in a GP-compatible form, 

\begin{equation}
    \label{e:moispar}
    I_i \leq 2w_{\text{cap}_i}t_{\text{cap}_i}\left(\frac{h_{\text{cap}_i}}{2}\right)^2.
\end{equation}

The weight of the spar per wing section $i$ is, 

\begin{equation}
    \label{e:sparmass}
    \Delta W_i \geq g[\rho_{\text{cfrp}} (2w_{\text{cap}_i}t_{\text{cap}_i} + 2t_{\mathrm{shear}_i}(h_{\mathrm{cap}_i} + 2t_{\mathrm{cap}_i})) + \rho_{\mathrm{foam}} w_{\mathrm{cap}_i}h_{\mathrm{cap}_i} ] \Delta y_i 
\end{equation}

An alternative spar configuration is a box spar.  
A similar concept to the cap spar, the box spar for composite materials has two layers of unidirectional carbon fiber on both the top and bottom surfaces, each layer separated by a foam core as shown in Figure~\ref{f:boxspar}.
This spar configuration was used by Solar Impulse.\cite{zephyr_spar}

\begin{figure}[h!]
	\begin{center}
	\includegraphics[width=0.9\textwidth]{boxspar.pdf}
    \caption{\textbf{Cross sectional view of a box spar.}}
	\label{f:boxspar}
	\end{center}
\end{figure}

Again using a first order, conservative approximation, assuming the spar thickness and core thicknesses are small, the moment of inertia can be expresses as

\begin{equation}
    \label{e:moispar}
    I \leq w_{\text{cap}_i}t_{\text{cap}_i}h_{\text{cap}_i}^2.
\end{equation}

The weight of the box spar per section $i$ is 

\begin{equation}
    \label{e:sparmass}
    \Delta W_i \geq g [\rho_{\text{cfrp}} (4w_{\text{cap}_i}t_{\text{cap}_i} + 2t_{\mathrm{shear}_i}(h_{\mathrm{cap}_i} + 4t_{\mathrm{cap}_i} + 2t_{\mathrm{core}_i})) + 2\rho_{\mathrm{foam}} w_{\mathrm{cap}_i}t_{\mathrm{core}_i}] \Delta y_i
\end{equation}

Additional geometric constraints imposed on the width and thickness for both the cap spar and box spar configurations.  
The total spar cap thickness cannot be greater than the thickness of the airfoil cross section, $\tau_t = 0.115$.  
The width of the spar cap is assumed no greater than 15\% of chord, $\tau_w = 0.15$. 

\begin{align}
    c(y)\tau_t &\geq h_{\text{cap}_i} + 2t_{\text{cap}_i} \\
    c(y)\tau_w &\geq w_{\text{cap}_i} 
\end{align}

Each spar is subjected to two loading cases: a standard g loading and a distributed gust load.  The spar must meet both strength and stiffness constraints.
The ultimate tensile strength for unidirectional carbon fiber is $\sigma_{\text{cfrp}} = 1700$ [MPa].\cite{cfprop}
The tip deflection is constrained to be less than 20\% of the half span, $\frac{w_{\text{max}}}{b/2} = 0.2$.

It seems clear from the moment of inertia and weight constraints that the box spar will have a better performance than the cap spar.  
Using the GP model as a method of comparing the two configurations, the difference in performance can be quantified. 
Table~\ref{t:spar} shows difference in important variables for the two configurations. 

\begin{longtable}{lccccccccccccc}
    \caption{Spar Configuration Comparison}\\
    \toprule
    \toprule
    \label{t:spar}
    Variable                & Cap Spar      & Box Spar      \\ \hline
    $W_{\mathrm{MTO}}$      & 176.2 [lbs]   &  160.7 [lbs]  \\
    $W_{\mathrm{wing}}$     & 46.1 [lbs]    & 40.9 [lbs]    \\
    $W_{\mathrm{spar}}$     & 10.8 [lbs]    & 8.85 [lbs]    \\
    $AR$                    & 25.5          & 26.6          \\
    $b$                     & 63 [ft]       &  61.3 [ft] \\
    \bottomrule
\end{longtable}

The solar aircraft model was solved twice, once with each respective spar configuration, each solve taking under 0.05 seconds.  
The results show that even a slight change in the spar configuration can result in a 9\% reduction in overall aircraft weight.  
The spar weight decreased only by 2 lbs, but the added strength of this configuration allowed the aspect ratio to increase, reducing induced drag and ultimately decreasing battery and aircraft weight.
Additional spar configurations, including tube spars or truss spars can also be evaluated using this methodology to quickly approximate their tradeoffs. 

\subsection{Battery Placement Trade Study}

The baseline configuration assumes that the batteries are placed inside the wing so that the wing is span loaded.  
Due to possible manufacturing and operating challenges associated with placing batteries in the wing, it may be advantageous to place the batteries in a central fuselage.  
Using the GP solar model, this trade study can be accomplished by assuming an elliptical fuselage shape with sufficient volume to hold the batteries

\begin{equation}
    \label{e:fusevol}
    \mathcal{V}_{\text{fuse}} \geq \frac{E_{\mathrm{batt}}} {EV_{\mathrm{batt}}}, 
\end{equation}

where $E_{\mathrm{batt}}$ is the total required battery energy to complete the mission, and $(EV_{\mathrm{batt}}) = 800 $ [Whr/l] is the battery energy density. 
The dimensions of the fuselage are constrained by

\begin{equation}
    \label{e:fusevol2}
    \mathcal{V}_{\text{fuse}} \leq \frac{4}{3}\pi \frac{l_{\text{fuse}}}{2}R_{\text{fuse}}^2
\end{equation}

where $l_{\text{fuse}}$ is the length of the fuselage and $R_{\text{fuse}}$ is the radius. Using the length and radius, the surface area can be calculated using Thomsen's approximation,\cite{ellipsoidSA}

\begin{equation}
    \label{e:fusesa}
    3 \left( \frac{S_{\text{fuse}}}{\pi} \right)^{1.6075} \geq 2(2l_{\text{fuse}}R_{\text{fuse}})^{1.6075} + (4R_{\text{fuse}}^2)^{1.6075}.
\end{equation}

The weight of the fuselage is constrained by

\begin{equation}
    \label{e:fuseweight}
    W_{\text{fuse}} \geq S_{\text{fuse}} \rho_{A_{\text{cfrp}}} g
\end{equation} 

where $\rho_{A_{\text{cfrp}}} = 0.0975$ [g/cm$^2$], or the area density of two plies of carbon fiber.\cite{cfply}  The surface area is also used to calculate the drag assuming a skin friction based drag model,

\begin{align}
    \label{e:fusedrag}
    D_{\text{fuse}} &\geq C_f k_{\text{fuse}} \frac{1}{2} \rho V^2 S_{\text{fuse}} \\
    C_f &\geq \frac{0.455}{Re^{0.3}}
\end{align}

where $k_{\text{fuse}}$ is the form factor approximated by\cite{raymer}

\begin{equation}
    \label{e:fuseform}
    k_{\text{fuse}} \geq 1 + \frac{60}{(l_{\text{fuse}}/2R_{\text{fuse}})^3} + \frac{(l_{\text{fuse}}/2R_{\text{fuse}})}{400}.
\end{equation}

Given these assumptions and the requirements and design parameters listed in Tables~\ref{t:mreqs} and~\ref{t:params}, the solar aircraft model is guaranteed infeasible with a single fuselage. Only by relaxing the latitude requirement to 23$^{\circ}$ latitude was a feasible solution reached.  Table~\ref{t:fuse} shows a comparison of the two configurations at a latitude requirement of 23$^{\circ}$. 

\begin{longtable}{lccccccccccccc}
    \caption{Battery Placement Trade at 23$^{\circ}$ Latitude}\\
    \toprule
    \toprule
    \label{t:fuse}
                            & \multicolumn{2}{c}{Battery Placement} \\
    Variable                & Wing          &  Fuselage     \\ \hline
    $MTOW$                  & 132.2 [lbs]   & 236.3 [lbs]   \\
    $W_{\mathrm{wing}}$     & 33.1 [lbs]    & 66.0 [lbs]    \\
    $W_{\mathrm{batt}}$     & 56.7 [lbs]    & 102.3 [lbs]   \\
    $AR$                    & 26.34         & 22.0          \\
    $b$                     & 54.5 [ft]     &  66.88 [ft]   \\
    $C_L$                   & 1.123         &  0.997        \\
    \bottomrule
\end{longtable}

As expected there is a large benefit to placing the batteries inside the wing, the aircraft weight being almost half of what it would be with a central fuselage to store the batteries.  
This suggests that a compromise between performance and manufacturing or operating constraints may be multiple smaller fuselages placed spanwise along the wing.  

One way to understand why these two configurations are so different is to look at the sensitivities to the structural design parameters of loading factor, maximum allowable tip deflection and maximum allowable stress. 
Table~\ref{t:fusesens} lists the sensitivities to these parameters for both configurations. 
A non-zero sensitivity means the constraint or constraints involving that parameter are active.  
For example, because the sensitivity to the standard g-loading factor is non-zero when the batteries are placed in the fuselage then the aircraft structure is sized by the g-loading case.  
For both configurations, the sensitivity to maximum allowable stress is zero implying that both aircraft structures are stiffness constrained. 
Similarly, because the maximum allowable tip deflection sensitivity is non-zero and the material strength sensitivity is zero then the wing structure for both configurations is stiffness constrained, not strength constrained. 

\begin{longtable}{lccccccccccccc}
    \caption{Sensitivity to Structural Parameters}\\
    \toprule
    \toprule
    \label{t:fusesens}
                                    & \multicolumn{2}{c}{Battery Placement} \\
    Variable                        & Wing    &  Fuselage   \\ \hline
    $N_g$                           & 0.0     & 0.312       \\
    $N_{\mathrm{gust}}$             & 0.286   & 0.0         \\
    $\frac{w_{\text{max}}}{b/2}$    &  -0.286 & -0.312      \\
    $\sigma_{\mathrm{CFRP}}$        & 0.0     & 0.0         \\
    \bottomrule
\end{longtable}

From these sensitivities it is understood that in addition to the extra weight and drag of the fuselage, the centralized battery placement causes decreased structural efficiency. 
This causes the aircraft to increase its structural weight and decrease its aspect ratio which causes a further drag increase.  
Both of these require more battery weight to close the energy cycle, creating a snowball effect that results in a large difference between the two configuration weight and sizes. 

\section{Resizing for Different Requirements}

Because solar flux and wind speed vary with both season and latitude, it is not clear how aircraft size and weight would vary for different seasonal and latitude requirements.  
For the seasonal requirement, time is centered on the summer solstice such that 6-month availability corresponds to availability between March 21st and September 21st. 
Again using the solar GP model described by Burton and Hoburg, the model was solved 13 times for different requirements in a total of 2.67 seconds.  
Figure~\ref{f:season} shows the results of this study. 
Returned solutions for 10 and 12 month availability at 28$^{\circ}$ latitude were infeasible and therefore not shown.  

\begin{figure}[h!]
	\begin{center}
	\includegraphics[width=0.75\textwidth]{season.pdf}
    \caption{\textbf{Resizing study for various latitude and seasonal requirements.}}
	\label{f:season}
	\end{center}
\end{figure}

This trade study shows that different latitude and seasonal requirements can have drastic effects on aircraft size.  
Using these results as a baseline, a family of solar aircraft of various sizes and capabilities could be estimated to fulfill operational requirements. 

\section{Tail Boom Flexibility}

The model described by Burton and Hoburg size the horizontal tail based on a reasonable tail volume coefficient.\cite{burton_solar_2017}  
However, for highly flexible aircraft, the horizontal tails effectiveness is decreased by the flexibility of the tail boom which causes a slight lag in the controls.\cite{warren_mech}
A simple physics-based model is derived to capture the correct trends shown by tail boom flexibility to improve model fidelity. 
The derivation begins with a definition of aircraft moment, 

\begin{align}
    \label{e:mcenter}
    M_{cg} &= M_{\text{w}} + (x_{cg} - x_{ac})L_{\text{w}} - l_{\text{h}} L_{\text{h}} \\
    \label{e:eq6}
    \frac{M_{cg}}{qSc} = C_m &= C_m + \frac{x_{cg} - x_{ac}}{c} C_{L_W} - V_{\text{h}} C_{L_{\text{h}}}
\end{align}

It is assumed that the tail boom's effective root location is at the wing's aerodynamic center $x_{ac}$, and that the tail's pitching moment about its own aerodynamic center is negligible. 

Using moment and lift approximations

\begin{align}
    C_{m} &= \text{constant} \\
    C_{L} &= C_{L_0} + m_{\text{w}} \alpha \\
    \label{e:eq10}
    C_{L_{\text{h}}} &= C_{L_{h_0}} + m_{\text{h}} \left[\left( 1 - \frac{d\epsilon}{d\alpha}\right) - \theta \right] + C_{L_{h_{\delta}}}\delta_e \\
    \label{e:mw}
    m_{\text{w}} &= \frac{2\pi}{1 + 2/AR} \\
    m_{\text{h}} &= \frac{2\pi}{1 + 2/AR_{\text{h}}}
\end{align}

where $\epsilon$ is the wing's downwash angle seen at the tail. Using a vortex approximation and neglecting taper effects, we can estimate

\begin{equation}
    \frac{d\epsilon}{d\alpha} = \frac{m_{\text{w}}}{4\pi} \frac{c}{l_{\text{h}}}
\end{equation}

so that \eqref{e:eq10} can be written as

\begin{align}
    C_{L_{\text{h}}} &= \mathcal{F}^{-1} m_{\text{h}} \left( 1 - \frac{d\epsilon}{d\alpha}\right) \alpha + \mathcal{F}^{-1} C_{L_{h_{\delta}}}(\delta_e - \delta_{e_0}) \\
    \mathcal{F} &= 1 + m_{\text{h}} \frac{qS_{\text{h}}l_{\text{h}}^2}{EI_0}(1-\frac{1}{2}k) 
\end{align}

where $\mathcal{F}$ is defined as the tail boom flexibility factor. 
Using the wing lift coefficient and the recast tail lift coefficient, the pitching moment is equation and derivative are given as,

\begin{align}
    C_m = C_{m_{\text{w}}} &+ \frac{x_{cg} - x_{ac}}{c} (C_{L_0} + m_{\text{w}} \alpha) \nonumber\\ 
    &- V_{\text{h}} \left[ \mathcal{F}^{-1} m_{\text{h}} \left( 1 - \frac{d\epsilon}{d\alpha}\right) \alpha + \mathcal{F}^{-1} C_{L_{h_{\delta}}}(\delta_e - \delta_{e_0})\right] \\
    \frac{dC_m}{d\alpha} & = \frac{x_{cg} - x_{ac}}{c} m_{\text{w}}  - V_{\text{h}} \mathcal{F}^{-1} m_{\text{h}} \left( 1 - \frac{d\epsilon}{d\alpha}\right).
\end{align}

Dividing by $dC_{L_W}/d\alpha = m_{\text{h}}$, 

\begin{equation}
    -\frac{dC_m/d\alpha}{dC_{L_W}/d\alpha} = \text{SM} = V_{\text{h}} \mathcal{F}^{-1} \frac{m_{\text{h}}}{m_{\text{w}}} \left( 1 - \frac{d\epsilon}{d\alpha}\right) - \frac{x_{cg} - x_{ac}}{c}.
\end{equation}

The case for meeting the minimum static margin requirement is at the never-exceed dynamic pressure $q_{NE}$ and when the c.g is at its aft-most position.

\begin{equation}
    \text{SM}_{\text{min}} = V_{\text{h}} \mathcal{F}_{NE}^{-1} \frac{m_{\text{h}}}{m_{\text{w}}} \left( 1 - \frac{d\epsilon}{d\alpha}\right) - \frac{(x_{cg})_{\text{aft}} - x_{ac}}{c} 
\end{equation}

Dividing Equation~\eqref{e:eq6} by $C_{L_W}$, gives a requirement on the tail lift coefficient required to achieve a pitch trim condition $C_m=1$

\begin{equation}
    0 = \frac{C_{m_{\text{w}}}}{C_{L_W}} + \frac{x_{cg} - x_{ac}}{c} - V_{\text{h}} \frac{C_{L_{\text{h}}}}{C_{L_W}}.
\end{equation}

The pitch authority requirement is that at the forward-most c.g. position and maximum lift with the tail lift coefficient equal to the most-negative allowable value $(C_{L_{\text{h}}})_{\text{min}}$. Combining the static margin and pitch authority requirement results in the horizontal tail sizing equation 

\begin{equation}
    \label{e:tbflex}
    \text{SM}_{\text{min}} + \frac{\Delta x_{cg}}{c} - \frac{C_{M_{\text{w}}}}{C_{L_{\text{max}}}} \leq V_{\text{h}} \mathcal{F}_{NE}^{-1} \frac{m_{\text{h}}}{m_{\text{w}}} \left( 1 - \frac{d\epsilon}{d\alpha}\right) + V_{\text{h}} \frac{-(C_{L_{\text{h}}})_{\text{min}}}{C_{L_{\text{max}}}}.
\end{equation}

where $\Delta x_{cg} = (x_{cg})_{\text{aft}} - (x_{cg})_{\text{fwd}}$. 
This set of equations is GP compatible with the exception of Equations~\ref{e:tbflex} and~\ref{e:mw}, which require signomial programming (SP) to solve.  
SP models solve a difference of convex program as described by Boyd.\cite{gp}

A comparison of the horizontal tail sizing by volume coefficient and by using the tail boom flexibility model show that incorporating the flexibility model into the aircraft model captures the correct design trends.  
Table~\ref{t:tbflex} shows relevant variable values with and without the tail boom flexibility model.  

\begin{longtable}{lccccccccccccc}
    \caption{Design Variables}\\
    \toprule
    \toprule
    \label{t:tbflex}
    Variable           & With Tail Flex     & Without Tail Flex \\ \hline
    $MTOW$             & 200.4 [lbs]        & 176.2 [lbs]       \\
    $V_{\mathrm{h}}$   & 0.577              & 0.45              \\
    $l_{\mathrm{h}}$   & 13.6 [ft]          & 12.6 [ft]         \\
    $S_{\mathrm{h}}$   & 19.6 [ft$^2$]      & 13.86 [ft$^2$]    \\
    $W_{\mathrm{h}}$   & 4.1 [lbs]          & 2.4 [lbs]         \\
    \bottomrule
\end{longtable}

As expected, a larger horizontal tail surface area and longer tail boom were required to account for the tail boom flexibility.  
This is an example of how simple physics-based models or surrogate models created by higher fidelity tools can be incorporated into a GP model to improve fidelity and capture important design trends.  

\section{Conclusion}

This paper leverages GP to perform design parameter, configuration, and mission requirement trade studies for solar-electric powered aircraft.  
These trade studies can be performed efficiently allowing for rapid analysis of a design space.  
The results are shown to capture important trends to guide conceptual and preliminary solar aircraft design. 
Specifically, it is observed that the box spars can significantly reduce aircraft weight.  
It is also shown through design sensitivities that improving battery technology will have a larger impact on aircraft performance that improving solar cell efficiency. 
A resizing study shows that much smaller solar aircraft can be designed by relaxing latitude and seasonal requirements. 
Finally, this paper demonstrates that higher order models can approximated by physics based models and fit into a GP form to produce meaningful results. \\


\emph{NOTE: This document is an abstract and is not a complete paper.  
For the final submitted paper we are planning to add more extensive model documentation and detailed results for configuration trade studies, sensitivity analysis and resizing calculations.}

\bibliography{biblibrary}
\bibliographystyle{aiaa}

\end{document}

